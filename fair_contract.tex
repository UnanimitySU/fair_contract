

\documentclass[a4paper,12pt]{article}

%%% Работа с русским языком
\usepackage{cmap}     % поиск в PDF
\usepackage{mathtext}     % русские буквы в фомулах
\usepackage[T2A]{fontenc}   % кодировка
\usepackage[utf8]{inputenc}   % кодировка исходного текста
\usepackage[english,russian]{babel} % локализация и переносы

%%% Работа с таблицами
\usepackage{array,tabularx,tabulary,booktabs} % Дополнительная работа с таблицами
\usepackage{longtable}  % Длинные таблицы
\usepackage{multirow} % Слияние строк в таблице
 
%%% Программирование
\usepackage{etoolbox} % логические операторы

%%% Страница
\usepackage{extsizes} % Возможность сделать 14-й шрифт
\usepackage{geometry} % Простой способ задавать поля
 \geometry{top=20mm}
 \geometry{bottom=35mm}
 \geometry{left=20mm}
 \geometry{right=20mm}
 %
\usepackage{fancyhdr} % Колонтитулы
  \pagestyle{fancy}
  \renewcommand{\headrulewidth}{0mm}  % Толщина линейки, отчеркивающей верхний колонтитул

  \lfoot{Подпись продавца}
  \rfoot{Подпись покупателя}
  \rhead{}
  \chead{}
  \lhead{}
  % \cfoot{Нижний в центре} % По умолчанию здесь номер страницы

\usepackage{setspace} % Интерлиньяж
%\onehalfspacing % Интерлиньяж 1.5
%\doublespacing % Интерлиньяж 2
%\singlespacing % Интерлиньяж 1

\usepackage{lastpage} % Узнать, сколько всего страниц в документе.

\usepackage{soul} % Модификаторы начертания

\usepackage{hyperref}
\usepackage[usenames,dvipsnames,svgnames,table,rgb]{xcolor}
\hypersetup{    % Гиперссылки
    unicode=true,           % русские буквы в раздела PDF
    pdftitle={Заголовок},   % Заголовок
    pdfauthor={Автор},      % Автор
    pdfsubject={Тема},      % Тема
    pdfcreator={Создатель}, % Создатель
    pdfproducer={Производитель}, % Производитель
    pdfkeywords={keyword1} {key2} {key3}, % Ключевые слова
    colorlinks=true,        % false: ссылки в рамках; true: цветные ссылки
    linkcolor=red,          % внутренние ссылки
    citecolor=green,        % на библиографию
    filecolor=magenta,      % на файлы
    urlcolor=cyan           % на URL
}

%\renewcommand{\familydefault}{\sfdefault} % Начертание шрифта

\usepackage{multicol} % Несколько колонок

\title{Договор купли-продажи автомобиля № 1}
\date{\today}

\begin{document} % конец преамбулы, начало документа

  \begin{center}
 \vspace{1 ex}
 \textbf{ДОГОВОР КУПЛИ-ПРОДАЖИ АВТОМОБИЛЯ № }
 \vspace{1ex}
    \end{center}
    
    \begin{flushright}01.01.2017\end{flushright}   \begin{flushleft}г.Москва\end{flushleft} 

 ООО «Три поросенка», именуемое в дальнейшем “Продавец”  в лице Иванова Ивана Ивановича действующего на основании  Доверенности № 75 от 01.09.2016 г. с  одной  стороны, и  г-н (г-жа), именуемый(ая) в дальнейшем "Покупатель", действующий(ая) от своего имени и в своих интересах, с другой стороны, далее совместно именуемые "Стороны", заключили настоящий Договор о нижеследующем:

\begin{center}
 \vspace{1 ex}
 \textbf{1.ПРЕДМЕТ ДОГОВОРА}
 \vspace{1ex}
    \end{center}
1.1. В соответствии с условиями настоящего Договора, Продавец принимает на себя обязательство передать в собственность Покупателя, а Покупатель обязуется принять и оплатить автомобиль: 

\begin{table}[h!]
 \caption{}
 \centering
  \begin{tabular}{|c|c|}
  \hline \verb|Марка, модель  ТС              | & \normalsize          \\
  \hline \verb|Идентификационный номер (VIN): | & \normalsize          \\
  \hline \verb|Год  изготовления:             | & \normalsize 2017 год \\
  \hline \verb|Тип автомобиля                 | &  \normalsize         \\
  \hline \verb|Модель, № двигателя:           | &  \normalsize         \\
  \hline \verb|Шасси (рама) №                 | &  \normalsize         \\
  \hline \verb|Кузов №                        | &  \normalsize         \\
  \hline \verb|Цвет кузова                    | &  \normalsize         \\
  \hline \verb|Паспорт ТС:                    | &  \normalsize         \\  
  \hline \verb|Выдан:                         | &  \normalsize         \\
  \hline \verb|Когда выдан:                   | &  \normalsize         \\
  \hline
  \end{tabular}
\end{table}

1.2. Комплектность автомобиля указана в Спецификации, согласованной сторонами и являющейся неотъемлемой частью настоящего договора (Приложение №1).

1.3. Данные идентификационного номера автомобиля, номер и дата выдачи ПТС, модель и номер двигателя, номер шасси и кузова фиксируются в Акте приёма передачи, (Приложение №2) являющимся неотъемлемой частью настоящего Договора.

1.4. Продавец гарантирует, что продаваемый автомобиль не заложен, не арестован, не обременен обязательствами перед третьими лицами, и что при его ввозе на территорию РФ соблюдены все необходимые таможенные формальности и уплачены все необходимые платежи, в соответствии с действующим Законодательством.

1.5. Право собственности на Автомобиль и риски случайной гибели и повреждения Автомобиля переходят от Продавца к Покупателю в момент передачи Автомобиля по Акту. С указанного момента Покупатель несет бремя ответственности за сохранность и целостность Автомобиля.

\newpage

\begin{center}
 \vspace{1 ex}
 \textbf{2.ЦЕНА АВТОМОБИЛЯ}
 \vspace{1ex}
    \end{center}
\begin{description}\item
2.1. Общая стоимость автомобиля согласована и установлена Сторонами в размере: стоимость-автомобиля-цифрами рублей
(стоимость-автомобиля-прописью рублей 00 копеек), в том числе НДС в размере сумма-НДС-цифрами (сумма-НДС-прописью)
\end{description}

\begin{description}\item
2.2. Цена Автомобиля, указанная в п. 2.1 настоящего Договора, помимо базовой стоимости включает в себя:\end{description}
 \begin{itemize}\item все налоговые, таможенные и иные обязательные платежи, связанные с ввозом Автомобиля на территорию РФ и его таможенным оформлением, именуемые обязательными платежами;\end{itemize}
 \begin{itemize}\item стоимость транспортных услуг по доставке Автомобиля до склада Продавца в г. Москве затраты по хранению Автомобиля на складе Продавца за исключением случаев, предусмотренных п.4.2. настоящего Договора;\end{itemize} 
 \begin{itemize}\item гарантийное обслуживание в объеме и на условиях, предусмотренных настоящим Договором.\end{itemize} 
  
\begin{description}\item2.3. Стороны согласились с тем, что в период исполнения обязательств по настоящему Договору,  Продавец оставляет за собой право корректировки общей стоимости  и сроков поставки автомобиля, а Покупатель исключает  предъявление Продавцу претензий, после предоставления ему соответствующих подтверждений в случаях:\end{description}
 \begin{itemize}\item изменения положений таможенного или иного законодательства Российской Федерации;\end{itemize}
 \begin{itemize}\item изменения условий заказа на заводе-изготовителе, влияющих на заводские цены и/или сроки поставки автомобиля;\end{itemize}
 \begin{itemize}\item изменения отпускных цен Производителем автомобилей THE-MARK-NAME в России.\end{itemize}
   
\begin{description}\item2.4. Оплата всех денежных платежей, указанных в настоящем договоре, осуществляется Покупателем в российских рублях.\end{description}


\newpage
\begin{center}
 \vspace{1 ex}
 \textbf{3.ПОРЯДОК ОПЛАТЫ РАСЧЁТОВ}
 \vspace{1ex}
    \end{center}
\begin{description}\item
3.1. Резерв автомобиля производится на основании внесения Покупателем аванса в размере 1\% от общей стоимости автомобиля.\end{description}
\begin{description}\item3.1.1. Стороны согласились с тем, что платеж, вносимый Покупателем может составлять меньшую сумму при условии внесения Покупателем остатка суммы в соответствии с п.3.1. в течении 2-х дней с момента осуществления первого платежа. В случае невыполнения Покупателем доплаты (п.3.1.) Продавец имеет право данный автомобиль реализовать.\end{description}
\begin{description}\item3.1.2. Оставшаяся неоплаченной часть в размере 99\% общей стоимости автомобиля оплачивается Покупателем в течение  5 (Пять) банковских дней со дня подписания настоящего договора. \end{description}
\begin{description}\item3.2. Датой оплаты автомобиля является дата поступления средств, в размере 100\% от общей стоимости автомобиля, на расчетный счет Продавца.
\end{description}
 
 \begin{center}
 \vspace{1 ex}
 \textbf{4. СРОКИ ПЕРЕДАЧИ }
 \vspace{1ex}
    \end{center}
\begin{description}\item4.1. Продавец принимает на себя обязательство передать Покупателю автомобиль в течение 15 (Пятнадцати) календарных дней с момента полной оплаты автомобиля. (п. 3.2.). \end{description}
\begin{description}\item4.2. Покупатель обязуется принять автомобиль в течение 5 (пяти) календарных дней с момента оповещения продавцом о готовности передать автомобиль. Каждый день хранения автомобиля сверх указанного срока оплачивается Покупателем из расчета 250 (Двести пятьдесят) рублей, за день хранения. \end{description}

\begin{description}\item4.3. О приемке Покупателем автомобиля составляется Акт приема-передачи, который подписывается представителями сторон. Стороны согласовали, что время для визуального осмотра кузовных деталей, а также проверки световых приборов, Продавцом при передаче автомобиля Покупателю, не ограничивалось. Осмотр, обслуженного в мойке, автомобиля проводился при нормальном освещении, позволяющим выявить все недостатки посредством визуального осмотра и проверки автомобиля, не требующих специальных навыков.\end{description}
\begin{description}\item4.4.  Стороны пришли к взаимному соглашению, что после подписания Акта о приёме автомобиля (Приложение №2), взаимные претензии по спецификации, комплектации, работе световых приборов (фары, стоп сигналы, и.т.д.), Сторонами исключаются.\end{description}
\begin{description}\item4.5. Покупатель согласился с тем, что в соответствии со ст. 10. ФЗ № 234 «О защите прав потребителей», Продавец своевременно и в полном объеме предоставил необходимую и достоверную информацию,  о приобретаемом товаре (п.1.1  Договора), обеспечивающую возможность его правильного выбора и надлежащей проверки.(п.п. 5.6.,5.7.Договора). \end{description}
 \newpage
 \begin{center}
 \vspace{1 ex}
 \textbf{5. КАЧЕСТВО ТРАНПОРТНОГО СРЕДСТВА И ГАРАНТИИ }
 \vspace{1ex}
    \end{center}

\begin{description}\item5.1. Качество автомобиля должно соответствовать стандартам, действующим в стране производителя, и/или техническим требованиям производителя.\end{description}
\begin{description}\item5.2. Автомобиль обеспечивается гарантией завода-изготовителя на условиях завода-изготовителя. Покупателю предоставляется гарантия качества Автомобиля на срок 36 (Тридцать шесть) месяцев или до достижения пробега Автомобиля 100 000 км с момента передачи Автомобиля Покупателю, в зависимости от того, что наступит ранее. Гарантия на Автомобиль предоставляется при условии соблюдения правил эксплуатации и управления, указанных в «Руководстве по эксплуатации» на автомобиль, в условиях настоящего Договора, а также своевременном прохождении технического обслуживания в любом авторизированном центре THE-MARK-NAME, что должно быть подтверждено соответствующими отметками в сервисной книжке автомобиля. \end{description}
На регулировочные работы, произведенные на заводе (в т.ч. схождение и развал колес), гарантийный срок ограничен 6 мес. или 10 тыс. пробега.
\begin{description}\item5.3. Гарантийный ремонт производится Сервисной службой, находящейся по адресу: адрес-сервисного-центра-дилера.
\end{description}
\begin{description}\item5.4. Предоставляемая Продавцом гарантия качества означает ответственность Продавца за недостатки качества и в зависимости от обстоятельств, предполагает замену или ремонт неисправных деталей Договорной продукции. Замененные по гарантии детали переходят в собственность Продавца.\end{description}
\begin{description}\item5.5. Гарантия качества распространяется на комплектующие изделия Автомобиля и считается равной гарантийному сроку на Автомобиль и истекает одновременно с истечением гарантийного срока на Автомобиль. Гарантия качества на комплектующие изделия автомобиля, замененные Продавцом в рамках гарантийных обязательств, предусмотренных настоящим Договором, истекает одновременно с истечением гарантийного срока на Автомобиль.\end{description}
\begin{description}\item5.6. Выполнение любых работ в рамках предпродажной подготовки направлено на передачу Покупателю качественного товара, основано на положениях законодательства и одобрено Покупателем при подписании настоящего Договора. Сам по себе факт проведения каких-либо работ в рамках предпродажной подготовки не может в дальнейшем рассматриваться как доказательство каких-либо недостатков принятого Покупателем Автомобиля.\end{description}
\begin{description}\item5.7. Гарантия утрачивает силу в следующих случаях:\end{description}
 \begin{itemize}\item при нарушении Покупателем условий эксплуатации Автомобиля, в частности, указанных в инструкции по его эксплуатации, при несоблюдении Покупателем требований, содержащихся в сервисной книжке, а также нарушении общеобязательных требований законодательства (в том числе, Правил дорожного движения), если это явилось причиной возникновения или увеличения дефекта;\end{itemize}
 \begin{itemize}\itemнепрохождения (или несвоевременного прохождения) инспекционного технического обслуживания на авторизованных станциях сервисного и технического обслуживания марки  THE-MARK-NAME в соответствии с требованиями сервисной книжки, а равно прохождение инспекционного технического обслуживания, выполнения ремонтных и иных работ на неавторизованных станциях сервисного и технического обслуживания, если это явилось причиной возникновения или увеличения дефекта.\end{itemize}
\begin{description}\item5.8. Гарантия качества ограничена только дефектами производственного характера и не распространяется на:\end{description}
 \begin{itemize}\itemтехнико-эксплуатационные регулировки Автомобиля, другие диагностические и регулировочные работы, связанные с естественным износом;\end{itemize}
 \begin{itemize}\itemестественный износ деталей, в том числе и ускоренный, если он вызван внешними воздействиями (дефектами дорожного покрытия, стилем вождения, условиями хранения и эксплуатации и др.), а равно несоблюдением рекомендаций, указанных в руководстве по эксплуатации Автомобиля;\end{itemize}
 \begin{itemize}\itemповреждения Автомобиля и любых его элементов, вызванные внешними воздействиями химических веществ, кислоты, частей дорожного покрытия, камней, песка, соли, пожаров, техногенной деятельностью человека, его небрежностью или неправомерными действиями, а также природными и экологическими явлениями (смолистые осадки деревьев, град, шторм, молнии, сильные ливни), и стихийными бедствиями;\end{itemize}
 \begin{itemize}\itemпроявляющиеся вследствие эксплуатации и являющиеся конструктивной особенностью Автомобиля незначительные шумы (щелчки, скрип, вибрация), не влияющие на качество, характеристики и работоспособность Автомобиля или его элементов, а также незначительное (не влияющее на нормальный расход) просачивание жидкостей сквозь прокладки и сальники, не различимые без применения специальных методов диагностики недостатки элементов отделки, лакокрасочного и гальванического покрытия;\end{itemize}
 \begin{itemize}\itemповреждения Автомобиля возникшие в результате дорожно-транспортного происшествия;\end{itemize}
 \begin{itemize}\itemустранение последствий ремонта (обслуживания), выполненного не уполномоченными производителем и/или импортером на проведение сервисного и технического обслуживания лицами;\end{itemize}
\begin{description}\item5.9. Гарантия качества не распространяется на недостатки и ущерб, возникшие в результате:\end{description}
 \begin{itemize}\itemнарушения правил эксплуатации и управления Автомобилем, которые описаны в сервисной книжке и Руководстве по эксплуатации (например, при несоблюдении требований к периодическому осмотру и инспекционному техническому обслуживанию на авторизованных станциях сервисного и технического обслуживания марки  THE-MARK-NAME.\end{itemize}
 \begin{itemize}\itemнеосторожного обращения с Автомобилем, перегрузок (например, в связи с использованием в спортивных целях, превышения допустимых нагрузок на ось);\end{itemize}
 \begin{itemize}\itemиспользования горюче-смазочных материалов и эксплуатационных жидкостей, не соответствующих характеристикам, указанным в руководстве по эксплуатации автомобиля;- установки и/или эксплуатации дополнительного оборудования и аксессуаров, которые не являются оригинальными оборудованием и аксессуарами марки  THE-MARK-NAME и/или,
 установки и/или эксплуатации дополнительного оборудования и аксессуаров, если такая установка выполнена иным способом, чем на авторизованной станции сервисного и технического обслуживания концерна  THE-MARK-NAME;\end{itemize}
\begin{description}\item5.10. Гарантийные обязательства не распространяются на следующие элементы и детали Автомобиля:\end{description}
 \begin{itemize}\itemрасходные и смазочные материалы, прочие элементы, используемые либо подверженные износу или разрушению при нормальной эксплуатации: воздушный, масляный и топливный фильтры, свечи зажигания, фрикционные материалы системы тормозов и сцепления, лампы накаливания, плавкие предохранители, рабочие жидкости и масла (масло, антифриз, тормозная жидкость, жидкость стеклоомывателя, хладагент системы воздушного кондиционирования), шины, щетки стеклоочистителей;\end{itemize}
\begin{description}\item5.11  Гарантию на изначально установленные на Автомобиле шины обеспечивает производитель данных шин. 
Если в процессе эксплуатации Автомобиля выявляется дефект материала, для получения компенсации следует обращаться к производителю шин напрямую.\end{description}
\begin{description}\item5.12 Стороны согласились с тем, что все установленное на продаваемый автомобиль дополнительное оборудование инициировано Покупателем и отражено в приложении к настоящему Договору в разделе «дополнительное оборудование» с гарантийным обязательством  Продавца,  сроком  на 12 месяцев с даты установки таких опций.\end{description}

 \begin{center}
 \vspace{1 ex}
 \textbf{6. ОТВЕТСТВЕННОСТЬ СТОРОН }
 \vspace{1ex}
    \end{center}

\begin{description}\item6.1. Покупатель вправе отказаться от исполнения договора в любое время при условии оплаты Продавцу фактически понесенных им расходов, связанных с исполнением обязательств по данному договору,  исходя из правил закреплённых в ст. 32 Закона РФ «О защите прав потребителей».\end{description}
\begin{description}\item6.2. В случае нарушения покупателем сроков оплаты стоимости автомобиля (п. 3.1. и 3.1.2. настоящего Договора), по истечении указанных сроков резерв на автомобиль автоматически снимается. В этом случае Продавец не гарантирует возможность покупки ранее зарезервированного автомобиля, а Покупатель не вправе предъявлять претензий, связанных с возможностью и сроками передачи ему автомобиля.\end{description} 
\begin{description}\item6.3. Продавец несет ответственность за качество продаваемого автомобиля. Если по качеству автомобиль не будет соответствовать техническим условиям, Продавец устраняет дефекты своими силами и за свой счет. Доставка автомобиля в Сервисный центр Продавца осуществляется Покупателем за свой счет.\end{description}
\begin{description}\item6.4. В случае невыполнения Продавцом  обязательств по настоящему договору (п.4.1) Покупатель имеет право получить компенсацию с Продавца по 0.1\%  за каждый день просрочки автомобиля, но не более 5\% от суммы указанной в п.2.1. настоящего Договора.
\end{description}
\begin{description}\item6.5. С момента передачи автомобиля по акту приема-передачи Покупателю переходят право собственности на автомобиль (автомобили) и все связанные с ним риски.\end{description}

 \begin{center}
 \vspace{1 ex}
 \textbf{7. ПРИМЕНИМОЕ ПРАВО И ПОРЯДОК РАССМОТРЕНИЯ СПОРОВ }
 \vspace{1ex}
    \end{center}

\begin{description}\item7.1. Настоящий Договор регулируется российским правом и подлежит толкованию в соответствии с ним.\end{description}
\begin{description}\item7.2. Все споры по настоящему Договору рассматриваются с соблюдением претензионного порядка. Претензии предъявляются в письменной форме и подписываются полномочным представителем Стороны, направляющей претензию. Претензия рассматривается в течение 10  (десяти) дней с момента ее получения. Сообщение о результатах рассмотрения претензии направляется заявителю в течение 3 (трех) дней, с даты окончания срока рассмотрения претензии. Претензии должны быть заявлены в письменном виде. К претензии должны быть приложены подтверждающие и обосновывающие её документы. Стороны согласовали, что срок рассмотрения претензии откладывается Продавцом на срок соразмерный сроку  представления Покупателем, полного документального подтверждения обстоятельств спора, изложенного в претензии и исчисляется с даты представления вышеизложенных документов Покупателем. \end{description}
\begin{description}\item7.3. Стороны пришли к взаимному соглашению, что в случае полного или частичного отказа в удовлетворении претензии или неполучении в срок (п. 7.2. настоящего Договора) ответа на претензию, заявитель обращается  за защитой своих прав в суд по месту территориальности заключения настоящего Договора, фактического осуществления коммерческой деятельности  Продавца. \end{description}

\begin{center}
 \vspace{1 ex}
 \textbf{8. ЗАЩИТА ПЕРСОНАЛЬНЫХ ДАННЫХ И ИСПОЛЬЗОВАНИЕ
ИНФОРМАЦИИ ПРОДАВЦОМ }
 \vspace{1ex}
    \end{center}

\begin{description}\item8.1. Покупатель выражает согласие и разрешает Продавцу обрабатывать свои персональные данные (фамилия, имя, отчество, год, месяц, дата и место рождения; адрес, номер паспорта и сведения о дате выдачи паспорта и выдавшем его органе; образование, профессия, место работы и должность; домашний, рабочий и мобильный телефоны; адрес электронной почты), включая сбор, систематизацию, накопление, хранение, уточнение (обновление, изменение) исключительно в целях сопровождения данного договора без права передачи третьим лицам. \end{description}
\begin{description}\item8.2. Данное Покупателем согласие на обработку его персональных данных действует в течение срока гарантийных обязательств Продавца и может быть отозвано посредством направления Покупателем письменного заявления на следующий почтовый адрес Продавца: 
\end{description}

\newpage
\begin{center}
 \vspace{1 ex}
 \textbf{9. ОСНОВАНИЯ  ОСВОБОЖДЕНИЯ ОТ ОТВЕТСТВЕННОСТИ }
 \vspace{1ex}
    \end{center}

\begin{description}\item9.1. Стороны настоящего Договора освобождаются от ответственности за частичное или полное неисполнение своих обязательств по Договору, если оно явилось следствием действия обстоятельств непреодолимой силы. Под обстоятельствами непреодолимой силы Стороны подразумевают внешние, чрезвычайные, непредсказуемые, непредотвратимые и непреодолимые события, которые не существовали в момент заключения Договора и возникшие помимо воли Покупателя и Продавца, в том числе: стихийные бедствия, землетрясения, наводнения, ураганы, пожары, технологические катастрофы, эпидемии, военные действия, противоправные действия третьих лиц, забастовки, правительственные меры и правовые нормативные акты, ограничивающие исполнение договорных обязательств. \end{description}

\begin{center}
 \vspace{1 ex}
 \textbf{10. ЗАКЛЮЧИТЕЛЬНЫЕ ПОЛОЖЕНИЯ }
 \vspace{1ex}
    \end{center}

\begin{description}\item10.1 Настоящий договор вступает в силу с момента его подписания и действует до момента полного выполнения Сторонами обязательств по настоящему Договору.\end{description}
\begin{description}\item10.2 Изменения и дополнения к настоящему Договору производятся только в письменном виде с согласия и дальнейшей подписи представителей Сторон.\end{description}
\begin{description}\item10.3 Настоящий Договор составлен в трех экземплярах, имеющих равную юридическую силу -  по одному для каждой из Сторон и один экземпляр для постановки автомобиля на учет в ГИБДД.\end{description}
\begin{description}\item10.4 Условия настоящего Договора и соглашений (протоколов и т.п.) к нему конфиденциальны и не подлежат разглашению.\end{description}
\begin{description}\item10.5 Продавец представил Покупателю необходимую и достоверную информацию и  в полном объёме, обеспечивающую возможность правильного выбора в соответствии с разделом № 1 настоящего Договора. \end{description}

\begin{center}
 \vspace{1 ex}
 \textbf{11. ЮРИДИЧЕСКИЕ АДРЕСА И РЕКВИЗИТЫ СТОРОН }
 \vspace{1ex}
    \end{center}


\begin{table}
\caption{\label{tab:canonsummary}ЮРИДИЧЕСКИЕ АДРЕСА И РЕКВИЗИТЫ СТОРОН.}
\begin{center}
\begin{tabular}{|c|c|}
\hline
“Продавец”  & “Покупатель” \\
\hline
ООО «Три поросенка»
 & Иванов Иван Иванович \\
Юридический адрес: & Адрес прописки  \\
 111111, Московская область, г. Москва\\ Западная общественная зона, \\шоссе Новаторов, дом №1 &  г. Москва, ул. Малых ленинцев,д.20,кв.20\\
\hline

\hline
р/с: р/с 00000000000000000000  (Руб)\\ в "Три-Банк" (ПАО) & Паспорт 11 11 № 111111 \\
БИК:111111111 & выдан 01.01.01 \\
к/c: 11111111111111111111 & Выдан: ОВД РАЙОНА ИВАНОВСКОЕ\\
КПП 100101001 & ГОР. МОСКВЫ \\
\hline
\end{tabular}
\end{center}
\end{table}

\end{document}
